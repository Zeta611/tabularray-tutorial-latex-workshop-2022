%! TEX program = xelatex
\documentclass{beamer}

\usepackage{kotex}

%%% Math settings
\usepackage{amssymb,amsmath} % Before unicode-math
\usepackage[math-style=TeX,bold-style=TeX]{unicode-math}

%% Font settings
\setmainfont{Libertinus Serif}[Scale=1.13]
\setsansfont{Libertinus Sans}[Scale=1.13]
\setmonofont{Inconsolata}[Scale=1.13]

\setmathfont{Libertinus Math}[Scale=1.13] % Before set*hangulfont

\setmainhangulfont{Noto Serif CJK KR}
\setsanshangulfont[BoldFont={* Bold}]{KoPubWorldDotum_Pro}
\setmonohangulfont{D2Coding}

%%%%%%%%%%%%%%%%%%%%%
%  Beamer Settings  %
%%%%%%%%%%%%%%%%%%%%%
\usetheme[numbering=fraction,progressbar=frametitle]{metropolis}
\useoutertheme[subsection=false]{miniframes}
\usecolortheme{rose}

\setbeamertemplate{itemize item}[square]
\setbeamertemplate{itemize subitem}[triangle]
\setbeamertemplate{itemize subsubitem}[circle]

% Custom commands
\usepackage{listings}
\lstdefinestyle{mystyle}{
    basicstyle=\ttfamily\scriptsize,
    breaklines=true
}
\lstset{style=mystyle}

\usepackage{hyperref}

\usepackage{tabularray}
\UseTblrLibrary{booktabs}
\UseTblrLibrary{siunitx}
\NewTableCommand\SC{\SetCell{bg=red8}}

\usepackage{tcolorbox}
\tcbuselibrary{listings,breakable}
\tcbset{listing engine=listings,colframe=black,colback=white,size=small}
\NewDocumentEnvironment{exampleside}{}%
  {\tcblisting{listing above text}}%
  {\endtcblisting}

\usepackage{chemmacros}

\newcommand*{\asgn}{\leftarrow}

\renewcommand*{\thefootnote}{\fnsymbol{footnote}}
\newcommand*{\manual}[1]{\texttt{Tabularray}\footnote[2]{버전 2022A (2022-03-01)} 매뉴얼 \textbf{#1}}


\title{\texttt{tabularray}로 표 그리기}
\author{이재호}
\date{2022년 5월 17일}

\begin{document}
\maketitle

\begin{frame}[t]
  \frametitle{목차}
  \tableofcontents
\end{frame}

% https://tex.stackexchange.com/search?page=2&tab=Relevance&q=user%3a106776%20%5btabularray%5d

\section{도입}
\begin{frame}[c,fragile]
  \frametitle{혜성처럼 등장한...}
  \begin{itemize}
    \item Overleaf users must download from \url{https://ctan.org/tex-archive/macros/latex/contrib/tabularray} and put it in the project directory.
    \item \verb/sudo tlmgr update tabularray/
    \item 2022A 기준
  \end{itemize}
\end{frame}

\begin{frame}[c]
  \frametitle{표를 그리자}
  \centering
  \begin{itemize}
    \item 가로선 및 세로선
    \item 색칠하기
    \item 정렬하기
    \item \ldots
  \end{itemize}

  \begin{tblr}{c|c}
    \toprule
    입력 & 출력 \\
    \midrule
    1 & 5 \\
    2 & 7 \\
    3 & \SC9 \\
    \bottomrule
  \end{tblr}
\end{frame}

\begin{frame}[c,fragile]
  \frametitle{간단한 표}
  \begin{columns}
    \begin{column}{0.5\textwidth}
      \begin{center}
        \begin{tblr}{|l|l|}
          \hline
          a & b \\
          \hline
          c & d \\
          \hline
        \end{tblr}
      \end{center}
    \end{column}

    \begin{column}{0.5\textwidth}
      \begin{lstlisting}
\begin{tblr}{|l|l|}
  \hline
  a & b \\
  \hline
  c & d \\
  \hline
\end{tblr}
      \end{lstlisting}
    \end{column}
  \end{columns}
\end{frame}


\section{기본}
\begin{frame}[c,fragile]
  \frametitle{열 정렬}
  \begin{columns}
    \begin{column}{0.35\textwidth}
      \begin{center}
        \begin{tblr}{|r|cl}
          \hline
          abc & {A\\B\\C} & C \\
          \hline
          $\alpha$ & 3.14 & \texttt{text} \\
          \hline
        \end{tblr}
      \end{center}
    \end{column}

    \begin{column}{0.65\textwidth}
      \begin{lstlisting}
\begin{tblr}{|r|cl}
  \hline
  abc & {A\\B\\C} & C \\
  \hline
  $\alpha$ & 3.14 & \texttt{text} \\
  \hline
\end{tblr}
      \end{lstlisting}
    \end{column}
  \end{columns}
\end{frame}

\begin{frame}[c,fragile]
  \frametitle{New Interface}
  \begin{columns}
    \begin{column}{0.35\textwidth}
      \begin{center}
        \begin{tblr}{
          hlines,
          vline{1,2} = {solid},
          colspec = {rcl},
        }
          abc & {A\\B\\C} & C \\
          $\alpha$ & 3.14 & \texttt{text} \\
        \end{tblr}
      \end{center}
    \end{column}

    \begin{column}{0.65\textwidth}
      \begin{lstlisting}
\begin{tblr}{
  hlines,
  vline{1,2} = {solid},
  colspec = {rcl},
}
  abc & {A\\B\\C} & C \\
  $\alpha$ & 3.14 & \texttt{text} \\
\end{tblr}
      \end{lstlisting}
    \end{column}
  \end{columns}
\end{frame}

\begin{frame}[c,fragile]
  \frametitle{열 타입}
  % 참고: \manual{???}

  정렬, 너비, 비율, 수식(!)까지

  \begin{columns}
    \begin{column}{0.5\textwidth}
      \begin{center}
        \begin{tblr}{
          vlines, hlines,
          colspec = {Q[$]Q[l,4em]},
        }
        a_i &   b   \\
            & text  \\
        x_i & text  \\
        \end{tblr}
      \end{center}
    \end{column}

    \begin{column}{0.5\textwidth}
      \begin{lstlisting}
\begin{tblr}{
  vlines, hlines,
  colspec = {Q[$]Q[l,4em]},
}
a_i &   b   \\
    & text  \\
x_i & text  \\
\end{tblr}
      \end{lstlisting}
    \end{column}
  \end{columns}
\end{frame}

\begin{frame}[c,fragile,allowframebreaks]
  \frametitle{행 정렬}
  % https://tex.stackexchange.com/a/597283/97583
  \begin{description}
    \item[h] text in the head of the cell
    \item[m] text in the middle
    \item[f] text in the foot of the cell
    \item[b] bottom line in the middle
    \item[t] top line in the middle
  \end{description}

  \begin{verbatim}
\begin{tblr}{
  Q[h,4em]Q[t,4em]Q[m,4em]Q[b,4em]Q[f,4em]
}
\begin{tblr}{h{4em}t{4em}m{4em}b{4em}f{4em}}
  \end{verbatim}

  % 참고: \manual{???}
  \framebreak
  \begin{center}
    \begin{tblr}{Q[h,4em]Q[t,4em]Q[m,4em]Q[b,4em]Q[f,4em]}
      \hline
      {row\\head} & {top\\line} & {middle} & {line\\bottom} & {row\\foot} \\
      \hline
      {row\\head} & {top\\line} & {11\\22\\mid\\44\\55} & {line\\bottom} & {row\\foot} \\
      \hline
    \end{tblr}
  \end{center}

  \framebreak
  \begin{lstlisting}
% \begin{tblr}{Q[h,4em]Q[t,4em]Q[m,4em]Q[b,4em]Q[f,4em]}
\begin{tblr}{h{4em}t{4em}m{4em}b{4em}f{4em}}
  \hline
  {row\\head} & {top\\line} & {middle} & {line\\bottom} & {row\\foot} \\
  \hline
  {row\\head} & {top\\line} & {11\\22\\mid\\44\\55} & {line\\bottom} & {row\\foot} \\
  \hline
\end{tblr}
  \end{lstlisting}
\end{frame}

\begin{frame}[c,fragile]
  \frametitle{행 타입}
  % 참고: \manual{???}

  \verb/\hline/까지 한 번에

  \begin{columns}
    \begin{column}{0.35\textwidth}
      \begin{center}
        \begin{tblr}{
          colspec = {Q[$]Q[l,4em]},
          rowspec = {|[2pt]Q[m]|[dotted]Q[b]|[dashed]Q[t]|},
        }
          a_i &   b   \\
              & text  \\
          x_i & text  \\
        \end{tblr}
      \end{center}
    \end{column}

    \begin{column}{0.65\textwidth}
      \begin{lstlisting}
\begin{tblr}{
  colspec = {Q[$]Q[l,4em]},
  rowspec = {|[2pt]Q[m]|[dotted]Q[b]|[dashed]Q[t]|},
}
  a_i &   b   \\
      & text  \\
  x_i & text  \\
\end{tblr}
      \end{lstlisting}
    \end{column}
  \end{columns}
\end{frame}

\begin{frame}[c,fragile]
  \frametitle{\texttt{verbatim}}
  \begin{columns}
    \begin{column}{0.5\textwidth}
      \begin{center}
        \begin{tblr}{verb}
          \verb/hello/ & \verb/world/ \\
          \verb/foo/ & \verb/bar/ \\
        \end{tblr}
      \end{center}
    \end{column}

    \begin{column}{0.5\textwidth}
      \begin{lstlisting}
\begin{tblr}{verb}
  \verb/hello/ & \verb/world/ \\
  \verb/foo/ & \verb/bar/ \\
\end{tblr}
      \end{lstlisting}
    \end{column}
  \end{columns}
\end{frame}

\section{심화}
\begin{frame}[c,fragile,allowframebreaks]
  \frametitle{열 너비 비율과 행 병합}
  \begin{itemize}
    \item \verb/X/ type column
    \item \verb/width=\linewidth/이 기본값
  \end{itemize}

  % 참고: \manual{???}

  \begin{center}
    \tiny
    \begin{tblr}{colspec={X[4,c]X[5,l]X[3,l]},rulesep=.8pt}
      \hline[1pt]
      Category & Operation & Description \\
      \hline
      \hline
      \SetCell[r=2]{m} {Arithmetic\\Operations} & Addition & $C \asgn A + B$ \\
                                  & Subtraction & $C \asgn A - B$ \\
      \hline
      \SetCell[r=6]{m} {Bitwise\\Boolean\\Operations} & NAND & $C \asgn \neg(A \mathbin{\&} B)$ \\
                                  & NOR & $C \asgn \neg(A \mathbin{|} B)$ \\
                                  & XNOR & $C \asgn A \mathbin{\equiv} B$ \\
                                  & AND & $C \asgn A \mathbin{\&} B$ \\
                                  & OR & $C \asgn A \mathbin{|} B$ \\
                                  & XOR & $C \asgn A \oplus B$ \\
      \hline
      \SetCell[r=4]{m} {Shifting\\Operations} & Logical Right Shift & $C \asgn A \gg 1$ \\
                                & Arithmetic Right Shift & $C \asgn A \ggg 1$ \\
                                & Logical Left Shift & $C \asgn A \ll 1$ \\
                                & Arithmetic Left Shift & $C \asgn A \lll 1$ \\
      \hline[1pt]
    \end{tblr}
  \end{center}

  \begin{lstlisting}
\begin{tblr}{colspec={X[4,c]X[5,l]X[3,l]},rulesep=.8pt}
  \hline[1pt]
  Category & Operation & Description \\
  \hline
  \hline
  \SetCell[r=2]{m} {Arithmetic\\Operations} & Addition & $C \asgn A + B$ \\
                              & Subtraction & $C \asgn A - B$ \\
  \hline
  \SetCell[r=6]{m} {Bitwise\\Boolean\\Operations} & NAND & $C \asgn \neg(A \mathbin{\&} B)$ \\
                              & NOR & $C \asgn \neg(A \mathbin{|} B)$ \\
                              & XNOR & $C \asgn A \mathbin{\equiv} B$ \\
                              & AND & $C \asgn A \mathbin{\&} B$ \\
                              & OR & $C \asgn A \mathbin{|} B$ \\
                              & XOR & $C \asgn A \oplus B$ \\
  \hline
  \SetCell[r=4]{m} {Shifting\\Operations} & Logical Right Shift & $C \asgn A \gg 1$ \\
                            & Arithmetic Right Shift & $C \asgn A \ggg 1$ \\
                            & Logical Left Shift & $C \asgn A \ll 1$ \\
                            & Arithmetic Left Shift & $C \asgn A \lll 1$ \\
  \hline[1pt]
\end{tblr}
  \end{lstlisting}
\end{frame}

\begin{frame}[c,fragile,allowframebreaks]
  \frametitle{문장과 정렬}
  % https://tex.stackexchange.com/a/608979/97583
  \verb/baseline/

  % 참고: \manual{???}

  \begin{center}
    Lorem ipsum
    \begin{tblr}[t]{hlines, colspec={c}, baseline=T}
    dolor \\ sit \\ amet,
    \end{tblr}
    consectetur
    \begin{tblr}[b]{hlines, colspec={c}, baseline=B}
    adipiscing \\ elit. \\
    \end{tblr}
  \end{center}

  \framebreak
  \begin{lstlisting}
Lorem ipsum
\begin{tblr}[t]{hlines, colspec={c}, baseline=T}
dolor \\ sit \\ amet,
\end{tblr}
consectetur
\begin{tblr}[b]{hlines, colspec={c}, baseline=B}
adipiscing \\ elit. \\
\end{tblr}
  \end{lstlisting}
\end{frame}


\begin{frame}[c,fragile]
  \frametitle{구분선 서식}
  \texttt{vline}, \texttt{hline}, \texttt{vlines} \& \texttt{hlines}

  % 참고: \manual{2.2 Hlines and Vlines}
  \begin{columns}
    \begin{column}{0.5\textwidth}
      \begin{center}
        \begin{tblr}{%
          hlines = {1}{-}{solid},
          vline{1,Z} = {1pt},
          vline{X} = {dashed},
        }
          A & B & C & D & E \\
          a & b & c & d & e \\
          0 & 1 & 2 & 3 & 4 \\
        \end{tblr}
      \end{center}
    \end{column}

    \begin{column}{0.5\textwidth}
      \begin{lstlisting}
\begin{tblr}{%
  hlines = {1}{-}{solid},
  vline{1,Z} = {1pt},
  vline{X} = {dashed},
}
  A & B & C & D & E \\
  a & b & c & d & e \\
  0 & 1 & 2 & 3 & 4 \\
\end{tblr}
      \end{lstlisting}
    \end{column}
  \end{columns}
\end{frame}

\begin{frame}[c,fragile,allowframebreaks]
  \frametitle{특별한 구분선}
  % \texttt{vline} \& \texttt{hline}

  참고: \manual{2.2 Hlines and Vlines}

  \begin{center}
    \begin{tblr}{%
      vlines, hlines,
      colspec = {lX[c]X[c]X[c]X[c]},
      vline{2} = {1}{ text = \clap{:} },
      vline{3} = {1}{ text = \clap{\ch{+}} },
      vline{4} = {1}{ text = \clap{\ch{->}} },
      vline{5} = {1}{ text = \clap{\ch{+}} },
    }
      Equation & \ch{CH4} & \ch{2 O2} & \ch{CO2} & \ch{2 H2O} \\
      Initial  & $n_1$    & $n_2$     & 0        & 0 \\
      Final    & $n_1-x$  & $n_2-2x$  & $x$      & $2x$ \\
    \end{tblr}
  \end{center}

  \framebreak
  \begin{lstlisting}
% \usepackage{chemmacros} in the preamble
\begin{tblr}{%
  vlines, hlines,
  colspec = {lX[c]X[c]X[c]X[c]},
  vline{2} = {1}{ text = \clap{:} },
  vline{3} = {1}{ text = \clap{\ch{+}} },
  vline{4} = {1}{ text = \clap{\ch{->}} },
  vline{5} = {1}{ text = \clap{\ch{+}} },
}
  Equation & \ch{CH4} & \ch{2 O2} & \ch{CO2} & \ch{2 H2O} \\
  Initial  & $n_1$    & $n_2$     & 0        & 0 \\
  Final    & $n_1-x$  & $n_2-2x$  & $x$      & $2x$ \\
\end{tblr}
  \end{lstlisting}
\end{frame}

\begin{frame}[c,fragile,allowframebreaks]
  \frametitle{표 명령어}
  % https://tex.stackexchange.com/a/630808/97583
  \begin{itemize}
    \item \verb/\NewTableCommand/를 사용하여 표 안에서 명령 사용 가능
    \item \verb/\NewTableCommand/ 안에는 텍스트 입력 불가
    \item \verb/expand/ 옵션으로 타협
  \end{itemize}

  참고: \manual{3.2.3 Expand Macros First, 3.6 New Table Commands}

  \begin{center}
    \begin{tblr}{%
        hlines, vlines,
        cell{1}{2-4} = {bg=gray9},
    }
      \SC 1 & Beta & Gamma & Delta    \\
      2 & Beta & Gamma & \SC Delta    \\
      3 & Beta & \SC Gamma & Delta    \\
    \end{tblr}
  \end{center}

  \framebreak
  \begin{lstlisting}
% \NewTableCommand\SC{\SetCell{bg=red8}} in the preamble
\begin{tblr}{%
    hlines, vlines,
    cell{1}{2-4} = {bg=gray9},
}
  \SC 1 & Beta & Gamma & Delta    \\
  2 & Beta & Gamma & \SC Delta    \\
  3 & Beta & \SC Gamma & Delta    \\
\end{tblr}
  \end{lstlisting}
\end{frame}

\begin{frame}[c,fragile,allowframebreaks]
  \frametitle{표 환경}
  % https://tex.stackexchange.com/a/634511/97583
  \begin{itemize}
    \item \verb/\NewTblrEnviron/를 사용하여 새로운 표 환경 정의 가능
    \item 내부 설정 (inner specifications)
    \item 외부 설정 (outer specifications)
  \end{itemize}

  참고: \manual{3.1 Inner Specifications, 3.2 Outer Specifications, 3.4 New Tabularray Environments, 3.5 New General Environments}

  \begin{center}
    \NewTblrEnviron{mytblr}
    \SetTblrInner[mytblr]{hlines,vlines}
    \SetTblrOuter[mytblr]{baseline=B}
      Text \begin{mytblr}{cccc}
      Alpha   & Beta  & Gamma  & Delta \\
      Epsilon & Zeta  & Eta    & Theta \\
      Iota    & Kappa & Lambda & Mu    \\
     \end{mytblr} Text
  \end{center}

  \framebreak
  \begin{lstlisting}
\NewTblrEnviron{mytblr}
\SetTblrInner[mytblr]{hlines,vlines}
\SetTblrOuter[mytblr]{baseline=B}
  Text \begin{mytblr}{cccc}
  Alpha   & Beta  & Gamma  & Delta \\
  Epsilon & Zeta  & Eta    & Theta \\
  Iota    & Kappa & Lambda & Mu    \\
 \end{mytblr} Text
  \end{lstlisting}
\end{frame}

\begin{frame}[c,fragile,allowframebreaks]
  \frametitle{siunitx + booktabs}

  % 참고: \manual{???}

  \begin{center}
    \tiny
    \begin{tblr}{S[table-format=4.2]S[table-format=3.3]S[table-format=3.3]S[table-format=3.3]}
      \toprule
      {{{$V_s$ [mV]}}} & {{{$V_R$ [mV]}}} & {{{$V_D$ [mV]}}} & {{{$I_D$ [µA]}}} \\
      \midrule
      100.47    & 0.035      & 100.002    & 0.035\\
      199.67    & 0.144      & 199.46     & 0.145\\
      296.41     & 0.974      & 295.17     & 0.983\\
      400.26     & 7.470      & 392.57     & 7.539\\
      500.13     & 36.904     & 463.48     & 37.247\\
      602.24     & 94.535     & 507.65     & 95.413\\
      703.92     & 168.73     & 535.10     & 170.30\\
      801.07     & 248.07     & 552.89     & 250.37\\
      899.73     & 332.25     & 567.62     & 335.34\\
      1008.14    & 428.76     & 579.35     & 432.74\\
      1104.33    & 514.13     & 589.49     & 518.90\\
      1200.23    & 602.45     & 597.38     & 608.04\\
      \bottomrule
    \end{tblr}
  \end{center}

  \framebreak
  \begin{lstlisting}
% \UseTblrLibrary{booktabs}
% \UseTblrLibrary{siunitx} in the preamble
\begin{tblr}{S[table-format=4.2]S[table-format=3.3]S[table-format=3.3]S[table-format=3.3]}
  \toprule
  {{{$V_s$ [mV]}}} & {{{$V_R$ [mV]}}} & {{{$V_D$ [mV]}}} & {{{$I_D$ [µA]}}} \\
  \midrule
  100.47    & 0.035      & 100.002    & 0.035\\
  199.67    & 0.144      & 199.46     & 0.145\\
  296.41     & 0.974      & 295.17     & 0.983\\
  400.26     & 7.470      & 392.57     & 7.539\\
  500.13     & 36.904     & 463.48     & 37.247\\
  602.24     & 94.535     & 507.65     & 95.413\\
  703.92     & 168.73     & 535.10     & 170.30\\
  801.07     & 248.07     & 552.89     & 250.37\\
  899.73     & 332.25     & 567.62     & 335.34\\
  1008.14    & 428.76     & 579.35     & 432.74\\
  1104.33    & 514.13     & 589.49     & 518.90\\
  1200.23    & 602.45     & 597.38     & 608.04\\
  \bottomrule
\end{tblr}
  \end{lstlisting}
\end{frame}

\begin{frame}[c]
  \begin{center}
    이외에도 다양하게 표를 꾸밀 수 있으니 매뉴얼을 참고해주세요.

    감사합니다.
  \end{center}
\end{frame}


% leftskip rightskip
% https://tex.stackexchange.com/questions/639150/tabularray-padding-per-cell/639181#639181

% @{}
% https://tex.stackexchange.com/a/638809/97583

\end{document}
