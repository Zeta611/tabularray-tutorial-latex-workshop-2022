%! TEX program = xelatex
\documentclass{beamer}

\usepackage{kotex}

%%% Math settings
\usepackage{amssymb,amsmath} % Before unicode-math
\usepackage[math-style=TeX,bold-style=TeX]{unicode-math}

\newfontfamily{\fallbackfont}{EB Garamond}
\DeclareTextFontCommand{\textfallback}{\fallbackfont}
\usepackage{newunicodechar}
\newunicodechar{⩴}{\textfallback{⩴}}

%% Font settings
\setmainfont{Libertinus Serif}[Scale=1.13]
\setsansfont{Libertinus Sans}[Scale=1.13]
\setmonofont{Inconsolata}[Scale=1.13]

\setmathfont{Libertinus Math}[Scale=1.13] % Before set*hangulfont

\setmainhangulfont{Noto Serif CJK KR}
\setsanshangulfont[BoldFont={* Bold}]{KoPubWorldDotum_Pro}
\setmonohangulfont{D2Coding}

%%%%%%%%%%%%%%%%%%%%%
%  Beamer Settings  %
%%%%%%%%%%%%%%%%%%%%%
\usetheme[numbering=fraction,progressbar=frametitle]{metropolis}
\useoutertheme[subsection=false]{miniframes}
\usecolortheme{rose}

\setbeamertemplate{itemize item}[square]
\setbeamertemplate{itemize subitem}[triangle]
\setbeamertemplate{itemize subsubitem}[circle]

% Custom commands
\usepackage{listings}
\lstdefinestyle{mystyle}{
    basicstyle=\ttfamily\scriptsize,
    breaklines=true
}
\lstset{style=mystyle}

\usepackage{hyperref}

\usepackage{tabularray}
\UseTblrLibrary{booktabs}
\UseTblrLibrary{siunitx}
\NewTableCommand\SC{\SetCell{bg=red8}}

\usepackage{tcolorbox}
\tcbuselibrary{listings,breakable}
\tcbset{listing engine=listings,colframe=black,colback=white,size=small}
\NewDocumentEnvironment{exampleside}{}%
  {\tcblisting{listing above text}}%
  {\endtcblisting}

\usepackage{chemmacros}

\renewcommand*{\thefootnote}{\fnsymbol{footnote}}
\newcommand*{\manual}[1]{\texttt{Tabularray}\footnote[2]{버전 2022A (2022-03-01)} 매뉴얼 \textbf{#1}}


\title{\texttt{tabularray}로 표 그리기}
\author{이재호}
\date{2022년 5월 17일}

\begin{document}
\maketitle

% \begin{frame}[t]
%   \frametitle{목차}
%   \tableofcontents
% \end{frame}

% https://tex.stackexchange.com/search?page=2&tab=Relevance&q=user%3a106776%20%5btabularray%5d

\section{도입}
\begin{frame}[c,fragile]
  \frametitle{혜성처럼 등장한...}
  \begin{itemize}
    \item Overleaf users must download from \url{https://ctan.org/tex-archive/macros/latex/contrib/tabularray} and put it in the project directory.
    \item \verb/sudo tlmgr update tabularray/
    \item 2022A 기준
  \end{itemize}
\end{frame}

\begin{frame}[c]
  \frametitle{표를 그리자}
  \begin{tblr}{cc}
    \toprule
    입력 & 출력 \\
    \midrule
    1 & 5 \\
    2 & 7 \\
    3 & \SC9 \\
    \bottomrule
  \end{tblr}
\end{frame}

\section{기본}
\begin{frame}[c,fragile,allowframebreaks]
  \frametitle{열 타입}
  % https://tex.stackexchange.com/a/608978/97583
  \texttt{colspec}

  참고: \manual{???}

  \begin{columns}
    \begin{column}{0.5\textwidth}
      \begin{center}
        \begin{tblr}{%
          vlines, hlines,
          colspec = {Q[$]l},
        }
        a_i &   b   \\
            & text  \\
        x_i & text  \\
        \end{tblr}
      \end{center}
    \end{column}

    \begin{column}{0.5\textwidth}
      \begin{lstlisting}
    \begin{tblr}{%
      vlines, hlines,
      colspec = {Q[$]l},
    }
      a_i &   b   \\
        & text  \\
      x_i & text  \\
    \end{tblr}
      \end{lstlisting}
    \end{column}
  \end{columns}
\end{frame}

\begin{frame}[c,fragile,allowframebreaks]
  \frametitle{칸 정렬}
  % https://tex.stackexchange.com/a/597283/97583
  \verb/tbhf/
  h: text in the head of the cell

  f: foot of the cell

  b: bottom line in the middle

  t: top line in the middle

  참고: \manual{???}

  \begin{center}
    \begin{tblr}[t]{hlines, colspec={c}, baseline=T}
      \hline
      {row\\head} & {top\\line} & {middle} & {line\\bottom} & {row\\foot} \\
      \hline
      {row\\head} & {top\\line} & {11\\22\\mid\\44\\55} & {line\\bottom} & {row\\foot} \\
      \hline
    \end{tblr}
  \end{center}

  \framebreak
  \begin{lstlisting}
\begin{tblr}[t]{hlines, colspec={c}, baseline=T}
  \hline
  {row\\head} & {top\\line} & {middle} & {line\\bottom} & {row\\foot} \\
  \hline
  {row\\head} & {top\\line} & {11\\22\\mid\\44\\55} & {line\\bottom} & {row\\foot} \\
  \hline
\end{tblr}
  \end{lstlisting}
\end{frame}

\begin{frame}[c,fragile,allowframebreaks]
  \frametitle{문장과 정렬}
  % https://tex.stackexchange.com/a/608979/97583
  \verb/baseline/

  참고: \manual{???}

  \begin{center}
    Lorem ipsum
    \begin{tblr}[t]{hlines, colspec={c}, baseline=T}
    dolor \\ sit \\ amet,
    \end{tblr}
    consectetur
    \begin{tblr}[b]{hlines, colspec={c}, baseline=B}
    adipiscing \\ elit. \\
    \end{tblr}
  \end{center}

  \framebreak
  \begin{lstlisting}
Lorem ipsum
\begin{tblr}[t]{hlines, colspec={c}, baseline=T}
dolor \\ sit \\ amet,
\end{tblr}
consectetur
\begin{tblr}[b]{hlines, colspec={c}, baseline=B}
adipiscing \\ elit. \\
\end{tblr}
  \end{lstlisting}
\end{frame}

\section{심화}
\begin{frame}[c,fragile,allowframebreaks]
  \frametitle{특별한 구분자}
  \texttt{vline} \& \texttt{hline}

  참고: \manual{2.2 Hlines and Vlines}

  \begin{center}
    \begin{tblr}{%
      vlines, hlines,
      colspec = {lX[c]X[c]X[c]X[c]},
      vline{2} = {1}{ text = \clap{:} },
      vline{3} = {1}{ text = \clap{\ch{+}} },
      vline{4} = {1}{ text = \clap{\ch{->}} },
      vline{5} = {1}{ text = \clap{\ch{+}} },
    }
      Equation & \ch{CH4} & \ch{2 O2} & \ch{CO2} & \ch{2 H2O} \\
      Initial  & $n_1$    & $n_2$     & 0        & 0 \\
      Final    & $n_1-x$  & $n_2-2x$  & $x$      & $2x$ \\
    \end{tblr}
  \end{center}

  \framebreak
  \begin{lstlisting}
% \usepackage{chemmacros} in the preamble
\begin{tblr}{%
  vlines, hlines,
  colspec = {lX[c]X[c]X[c]X[c]},
  vline{2} = {1}{ text = \clap{:} },
  vline{3} = {1}{ text = \clap{\ch{+}} },
  vline{4} = {1}{ text = \clap{\ch{->}} },
  vline{5} = {1}{ text = \clap{\ch{+}} },
}
  Equation & \ch{CH4} & \ch{2 O2} & \ch{CO2} & \ch{2 H2O} \\
  Initial  & $n_1$    & $n_2$     & 0        & 0 \\
  Final    & $n_1-x$  & $n_2-2x$  & $x$      & $2x$ \\
\end{tblr}
  \end{lstlisting}
\end{frame}

\begin{frame}[c,fragile,allowframebreaks]
  \frametitle{표 명령어}
  % https://tex.stackexchange.com/a/630808/97583
  \begin{itemize}
    \item \verb/\NewTableCommand/를 사용하여 표 안에서 명령 사용 가능
    \item \verb/\NewTableCommand/ 안에는 텍스트 입력 불가
    \item \verb/expand/ 옵션으로 타협
  \end{itemize}

  참고: \manual{3.2.3 Expand Macros First, 3.6 New Table Commands}

  \begin{center}
    \begin{tblr}{%
        hlines, vlines,
        cell{1}{2-4} = {bg=gray9},
    }
      1 & Beta & Gamma & Delta    \\
      2 & Beta & Gamma & Delta    \\
      3 & \SC Beta & \SC Gamma & Delta    \\
    \end{tblr}
  \end{center}

  \framebreak
  \begin{lstlisting}
% \NewTableCommand\SC{\SetCell{bg=red8}} in the preamble
  \begin{tblr}{%
      hlines, vlines,
      cell{1}{2-4} = {bg=gray9},
  }
    1 & Beta & Gamma & Delta    \\
    2 & Beta & Gamma & Delta    \\
    3 & \SC Beta & \SC Gamma & Delta    \\
  \end{tblr}
  \end{lstlisting}
\end{frame}

\begin{frame}[c,fragile,allowframebreaks]
  \frametitle{표 환경}
  % https://tex.stackexchange.com/a/634511/97583
  \begin{itemize}
    \item \verb/\NewTblrEnviron/를 사용하여 새로운 표 환경 정의 가능
    \item 내부 설정 (inner specifications)
    \item 외부 설정 (outer specifications)
  \end{itemize}

  참고: \manual{3.1 Inner Specifications, 3.2 Outer Specifications, 3.4 New Tabularray Environments, 3.5 New General Environments}

  \begin{center}
    \begin{tblr}{%
        hlines, vlines,
        cell{1}{2-4} = {bg=gray9},
    }
      1 & Beta & Gamma & Delta    \\
      2 & Beta & Gamma & Delta    \\
      3 & \SC Beta & \SC Gamma & Delta    \\
    \end{tblr}
  \end{center}

  \framebreak
  \begin{lstlisting}
% \NewTableCommand\SC{\SetCell{bg=red8}} in the preamble
  \begin{tblr}{%
      hlines, vlines,
      cell{1}{2-4} = {bg=gray9},
  }
    1 & Beta & Gamma & Delta    \\
    2 & Beta & Gamma & Delta    \\
    3 & \SC Beta & \SC Gamma & Delta    \\
  \end{tblr}
  \end{lstlisting}
\end{frame}

\begin{frame}[c,fragile,allowframebreaks]
  \frametitle{siunitx}

  참고: \manual{???}

  \begin{center}
    \tiny
    \begin{tblr}{S[table-format=4.2]S[table-format=3.3]S[table-format=3.3]S[table-format=3.3]}
      \toprule
      {{{$V_s$ [mV]}}} & {{{$V_R$ [mV]}}} & {{{$V_D$ [mV]}}} & {{{$I_D$ [µA]}}} \\
      \midrule
      100.47    & 0.035      & 100.002    & 0.035\\
      199.67    & 0.144      & 199.46     & 0.145\\
      296.41     & 0.974      & 295.17     & 0.983\\
      400.26     & 7.470      & 392.57     & 7.539\\
      500.13     & 36.904     & 463.48     & 37.247\\
      602.24     & 94.535     & 507.65     & 95.413\\
      703.92     & 168.73     & 535.10     & 170.30\\
      801.07     & 248.07     & 552.89     & 250.37\\
      899.73     & 332.25     & 567.62     & 335.34\\
      1008.14    & 428.76     & 579.35     & 432.74\\
      1104.33    & 514.13     & 589.49     & 518.90\\
      1200.23    & 602.45     & 597.38     & 608.04\\
      \bottomrule
    \end{tblr}
  \end{center}

  \framebreak
  \begin{lstlisting}
% \UseTblrLibrary{booktabs}
% \UseTblrLibrary{siunitx} in the preamble
\begin{tblr}{S[table-format=4.2]S[table-format=3.3]S[table-format=3.3]S[table-format=3.3]}
  \toprule
  {{{$V_s$ [mV]}}} & {{{$V_R$ [mV]}}} & {{{$V_D$ [mV]}}} & {{{$I_D$ [µA]}}} \\
  \midrule
  100.47    & 0.035      & 100.002    & 0.035\\
  199.67    & 0.144      & 199.46     & 0.145\\
  296.41     & 0.974      & 295.17     & 0.983\\
  400.26     & 7.470      & 392.57     & 7.539\\
  500.13     & 36.904     & 463.48     & 37.247\\
  602.24     & 94.535     & 507.65     & 95.413\\
  703.92     & 168.73     & 535.10     & 170.30\\
  801.07     & 248.07     & 552.89     & 250.37\\
  899.73     & 332.25     & 567.62     & 335.34\\
  1008.14    & 428.76     & 579.35     & 432.74\\
  1104.33    & 514.13     & 589.49     & 518.90\\
  1200.23    & 602.45     & 597.38     & 608.04\\
  \bottomrule
\end{tblr}
  \end{lstlisting}
\end{frame}


% leftskip rightskip
% https://tex.stackexchange.com/questions/639150/tabularray-padding-per-cell/639181#639181

% @{}
% https://tex.stackexchange.com/a/638809/97583

\end{document}
